%%%
% File: /latex/big-cocluster-paper/sections/experiment.tex
% Created Date: Tuesday, January 23rd 2024
% Author: Zihan
% -----
% Last Modified: Tuesday, 23rd January 2024 11:32:44 am
% Modified By: the developer formerly known as Zihan at <wzh4464@gmail.com>
% -----
% HISTORY:
% Date      		By   	Comments
% ----------		------	---------------------------------------------------------
%%%

\section{Experiments}
\label{sec:experiment}

\subsection{Datasets}

Here we describe the datasets used in our experiments.

\subsubsection{Amazon 1000}

This dataset contains 1000 Amazon reviews, each of which is represented by a 1000-dimensional vector. The dataset is available at \url{http://www.cs.utexas.edu/users/ml/riddle/data.html}.
% \begin{table}[htbp]
%     \centering
%     \caption{Statistics of Amazon 1000 dataset.}
%     \begin{tabular}{@{} l c c c @{}}
%         \toprule
%         Dataset     & \# of rows & \# of columns & \# of nonzeros \\
%         \midrule
%         Amazon 1000 & 1000       & 1000          & 1000000        \\
%         \bottomrule
%     \end{tabular}
% \end{table}

\subsubsection{CLASSIC4}

This dataset contains 18000 documents from 20 newsgroups, each of which is represented by a 1000-dimensional vector. The dataset is available at \url{http://www.cs.utexas.edu/users/ml/riddle/data.html}.


\begin{table}[htbp]
    \centering
    \caption{Running time of co-clustering methods on Amazon 1000, CLASSIC4, and RCV1-Large datasets.}
    \begin{tabular}{@{} l ccccc @{}}
        \toprule
        Dataset     & SCC (s) & DeepCC & PNMTF (s) & P-SCC (s) & P-PNMTF (s) \\
        \midrule
        Amazon 1000 & 64545.2 & *      & 303.7     & 112.5     & 242.8       \\
        CLASSIC4    & *       & *      & 17,810    & 22,894    & 3,028       \\
        RCV1-Large  & *       & *      & 277,092   & *         & 208,048     \\
        % ... more rows here
        \bottomrule
    \end{tabular}
    \begin{tablenotes}
        \small
        \item Notes: * indicates that the method can't deal with the dataset.
    \end{tablenotes}
\end{table}

\begin{table}[htbp]
    \centering
    \caption{NMIs and ARIs of co-clustering results on Amazon 1000, CLASSIC4, and RCV1-Large datasets.}
    \begin{tabular}{@{} l c ccccc @{}}
        \toprule
        \multirow{2}{*}{Dataset}    & \multirow{2}{*}{Metric} & \multicolumn{5}{c}{Comparison Methods}                                                         \\
        \cmidrule{3-7}
                                    &                         & SCC                                    & DeepCC & PNMTF  & Partitioned SCC & Partitioned PNMTF \\
        \midrule
        \multirow{2}{*}{Amazon}     & NMI                     & 0.9223                                 & *      & 0.6894 & 0.8650          & 0.6609            \\
                                    & ARI                     & 0.7713                                 & *      & 0.6188 & 0.7763          & 0.6057            \\
        \multirow{2}{*}{CLASSIC4}   & NMI                     & *                                      & *      & 0.5944 & 0.7676          & 0.6073            \\
                                    & ARI                     & *                                      & *      & 0.4523 & 0.5845          & 0.4469            \\
        \multirow{2}{*}{RCV1-Large} & NMI                     & *                                      & *      & 0.6519 & 0.8349          & 0.6348            \\
                                    & ARI                     & *                                      & *      & 0.5383 & 0.7576          & 0.5298            \\
        % ... more rows here
        \bottomrule
    \end{tabular}
    \begin{tablenotes}
        \small
        \item Notes: * indicates that the method can't deal with the dataset.
    \end{tablenotes}
\end{table}