% arara: latexmk: { options: ["-cd", "-outdir=../build/cover_letter"] }

%%%
% File: ./cover_letter.tex
% Created Date: Tuesday, February 11th, 2025
% Author: Zihan Wu
%%%
\documentclass[11pt]{letter}
\usepackage{hyperref}
\usepackage{geometry}
\geometry{margin=1in}
\signature{Zihan Wu\\
Department of Electrical Engineering\\
City University of Hong Kong\\
Hong Kong\\
Email: zihan.wu@my.cityu.edu.hk}
\begin{document}
\begin{letter}{Editor-in-Chief\\
IEEE Transactions on Systems, Man, and Cybernetics: Systems}

\opening{Dear Editor,}

We are pleased to submit our second revised manuscript titled ``DiMergeCo: A Scalable Framework for Large-Scale Co-Clustering with Theoretical Guarantees'' to \emph{IEEE Transactions on Systems, Man, and Cybernetics: Systems}, following the second round of review (Manuscript ID: SMCA-25-02-0634).

We sincerely thank the Senior Editor, Associate Editor, and reviewers for their continued constructive feedback. Their specific, actionable comments have led to substantial improvements in the theoretical rigor, experimental methodology, and clarity of our manuscript. We have carefully addressed all seven reviewer comments through the following revisions.

\textbf{Major Revisions in This Round:}

\begin{enumerate}
\item \textbf{Conditional Merging Quality Bound (Reviewer 2, Comment 1):} We have closed the theoretical gap identified in the merging step by introducing a new Conditional Merging Quality Bound (Theorem 8). This theorem provides the first formal guarantee for hierarchical merging under verifiable conditions, with a full proof in the supplementary material. An ablation study comparing five merging strategies across three datasets provides empirical validation.

\item \textbf{Statistical Reporting (Reviewer 1, Comment 1):} All experimental results in Table~I are now reported as mean $\pm$ standard deviation over 10 independent random seeds, with paired $t$-tests confirming statistical significance ($p < 0.01$).

\item \textbf{Scalability Comparison with PNMTF (Reviewer 1, Comment 2):} A new figure directly compares the wall-clock time of DiMergeCo-PNMTF versus PNMTF as a function of processing nodes, demonstrating the $O(\log n)$ communication advantage.

\item \textbf{Memory Profiling (Reviewer 1, Comment 3):} A new table reports peak memory usage (MB) for all methods across all datasets, showing that DiMergeCo reduces per-node memory by 5--8$\times$ on large datasets.

\item \textbf{Amazon 1000 Clarification (Reviewer 1, Comment 4):} A footnote now explicitly defines ``Amazon 1000'' as a 1,000-document subset used for controlled scalability profiling.

\item \textbf{Notation Unification (Reviewer 2, Comment 2):} All notation has been systematically unified: $\mathbf{A}$ for the data matrix and $C_k$ (non-bold) for co-clusters throughout both the text and the notation table.

\item \textbf{Modern Baseline (Reviewer 2, Comment 3):} SpectralCoclustering from scikit-learn has been added to all experimental tables, with a Related Work paragraph discussing modern baseline selection.
\end{enumerate}

\textbf{Summary of Contributions:}

This manuscript presents DiMergeCo, the first scalable co-clustering framework with formal theoretical guarantees spanning the entire distributed pipeline---from probabilistic partitioning through hierarchical merging. The framework preserves over 94\% of the base algorithm's clustering quality while scaling to datasets with 685K samples that centralized methods cannot process, outperforming existing methods in both scalability and accuracy across text, recommendation, and genomics domains.

The detailed Authors' Responses document systematically addresses each reviewer comment with specific manuscript modifications highlighted in blue. We are confident that the revised manuscript now meets the journal's standards for publication.

We appreciate your consideration of our revised submission and look forward to your feedback.

\closing{Best regards,}

\end{letter}
\end{document}
